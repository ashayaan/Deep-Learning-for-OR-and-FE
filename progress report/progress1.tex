\documentclass[12pt]{article}
\usepackage[T1]{fontenc}
\usepackage[utf8]{inputenc}
\usepackage{lmodern, textcomp} %textcomp for usage of euro currency symbol
\usepackage{listings}
\usepackage[margin=1in]{geometry}
\usepackage{csquotes}
\usepackage[english]{babel}
\usepackage{color}
\usepackage{xcolor} %for more colours to use in hyperref
\usepackage{amsmath}
\usepackage{makecell} %for resizing hline
\usepackage{float}
\usepackage{graphicx} %for pictures
\graphicspath{ {figures/} }
\usepackage[
    backend=biber,
    style=numeric,
    sorting=none
    ]{biblatex}
\addbibresource{ref.bib}

\usepackage{hyperref}
\hypersetup{
    colorlinks=true, %set true if you want colored links
    linkcolor={red!50!black},
    citecolor={blue!50!black},
    urlcolor={blue!80!black}
    }
    
\usepackage{listings}
\usepackage{color}


\usepackage{graphicx}
\usepackage{caption}
\usepackage{subcaption}

\definecolor{mygreen}{rgb}{0,0.6,0}
\definecolor{mygray}{rgb}{0.5,0.5,0.5}
\definecolor{mymauve}{rgb}{0.58,0,0.82}

\lstset{ %
  backgroundcolor=\color{white},   % choose the background color; you must add \usepackage{color} or \usepackage{xcolor}; should come as last argument
  basicstyle=\footnotesize,        % the size of the fonts that are used for the code
  breakatwhitespace=false,         % sets if automatic breaks should only happen at whitespace
  breaklines=true,                 % sets automatic line breaking
  captionpos=b,                    % sets the caption-position to bottom
  commentstyle=\color{mygreen},    % comment style
  deletekeywords={...},            % if you want to delete keywords from the given language
  escapeinside={\%*}{*)},          % if you want to add LaTeX within your code
  extendedchars=true,              % lets you use non-ASCII characters; for 8-bits encodings only, does not work with UTF-8
  frame=single,	                   % adds a frame around the code
  keepspaces=true,                 % keeps spaces in text, useful for keeping indentation of code (possibly needs columns=flexible)
  keywordstyle=\color{blue},       % keyword style
  language=Octave,                 % the language of the code
  morekeywords={*,...},           % if you want to add more keywords to the set
  numbers=left,                    % where to put the line-numbers; possible values are (none, left, right)
  numbersep=5pt,                   % how far the line-numbers are from the code
  numberstyle=\tiny\color{mygray}, % the style that is used for the line-numbers
  rulecolor=\color{black},         % if not set, the frame-color may be changed on line-breaks within not-black text (e.g. comments (green here))
  showspaces=false,                % show spaces everywhere adding particular underscores; it overrides 'showstringspaces'
  showstringspaces=false,          % underline spaces within strings only
  showtabs=false,                  % show tabs within strings adding particular underscores
  stepnumber=2,                    % the step between two line-numbers. If it's 1, each line will be numbered
  stringstyle=\color{mymauve},     % string literal style
  tabsize=2,	                   % sets default tabsize to 2 spaces
  title=\lstname                   % show the filename of files included with \lstinputlisting; also try caption instead of title
}

\usepackage{algorithm}
\usepackage[noend]{algpseudocode}
\makeatletter
\def\BState{\State\hskip-\ALG@thistlm}
\makeatother

\newcommand*\samethanks[1][\value{footnote}]{\footnotemark[#1]}

\title{\Large{\textbf{Progress Report}}\\\Large{IEORE 4742 Deep Learning for OR and FE}}

\author{
    Ahmad Shayaan \\as5948
    } 

\date{
\{\href{mailto:ahmad.shayaan@columbia.edu}{\texttt{\small{ahmad.shayaan@columbia.edu}}}\}\texttt{\small{@columbia.edu}}\\
    Columbia University\\
    \today}



\usepackage[utf8]{inputenc}

% Default fixed font does not support bold face
\DeclareFixedFont{\ttb}{T1}{txtt}{bx}{n}{12} % for bold
\DeclareFixedFont{\ttm}{T1}{txtt}{m}{n}{12}  % for normal

% Custom colors
\usepackage{color}
\definecolor{deepblue}{rgb}{0,0,0.5}
\definecolor{deepred}{rgb}{0.6,0,0}
\definecolor{deepgreen}{rgb}{0,0.5,0}

\usepackage{listings}

% Python style for highlighting
\newcommand\pythonstyle{\lstset{
		language=Python,
		basicstyle=\ttm,
		otherkeywords={self},             % Add keywords here
		keywordstyle=\ttb\color{deepblue},
		emph={MyClass,__init__},          % Custom highlighting
		emphstyle=\ttb\color{deepred},    % Custom highlighting style
		stringstyle=\color{deepgreen},
		frame=tb,                         % Any extra options here
		showstringspaces=false            % 
	}}
	
	
	% Python environment
	\lstnewenvironment{python}[1][]
	{
		\pythonstyle
		\lstset{#1}
	}
	{}
	
	% Python for external files
	\newcommand\pythonexternal[2][]{{
			\pythonstyle
			\lstinputlisting[#1]{#2}}}
	
	% Python for inline
	\newcommand\pythoninline[1]{{\pythonstyle\lstinline!#1!}}

\begin{document}

\maketitle
%\tableofcontents

\pagebreak

\section*{Problem Statement}
To train a Generative Adversarial Network with regularization parameter between layers so that the model learns to generate images with connected components.

\section*{Work done}

We have come up with various training paradigms to add regularization parameters at different levels in the generator architecture. We experimented with regularization at every layer, regularization at alternate layers and regularization at just the final layer. The function that we used as the regularization parameter is as follows.

\

\begin{python}
layer_n = tf.nn.sigmoid(layer_n-1)
layer_n = tf.clip_by_value(layer_n, clip_value_min = 0.5, clip_value_max=1)
layer_n = 1 - layer_n
\end{python}

\noindent The architecture consists of five convolution layers in both the generator and the discriminator. When we were regularization at every layer the model was not able to learn anything and generated garbage images. This was because we were being to aggressive in our regularization strategy. The image generated from the network with regularization at every layer are shown below.

\begin{figure}[H]
	\centering
	\includegraphics[width=7.5cm,height=7.5cm]{img2}
	\caption{Regularization at every layer}
\end{figure}

We then tried adding the regularization parameter at every other layer. The images generated by the network after training are shown below.

\begin{figure}[H]
	\centering
	\includegraphics[width=7.5cm,height=7.5cm]{img3}
	\caption{Regularization at every other layer}
\end{figure}
The images that have been generated are able to capture some structure in the sense that they contain a few connected components. But there are still a few black dots in the connected components. We also tried to add regularization parameter at only the last layer. The images generated by the trained network are shown below.

\begin{figure}[H]
	\centering
	\includegraphics[width=7.5cm,height=7.5cm]{img4}
	\caption{Regularization at only the last layer}
\end{figure}

The images generated by the network with regularization at only the last layer are worse than the images generated by the network with regularization at every other layer.
\\
\\
We are now working on designing new regularization function that will be able to better capture the complexity of the images.
\end{document}